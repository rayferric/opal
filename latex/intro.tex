\section{Introduction}

\subsection{What OPAL is and What it is not}

OPAL stands for Overpowered Algorithm Library. It is a collection of algorithms and data structures packaged as a header-only library for easy use with C++ 20 and higher. OPAL, however, is not a replacement for the standard library. It is not meant to replace any of the standard components as it is not optimized for speed in every way possible. It should rather be regarded as a learning tool and reference implementation for people who want to learn about algorithms and data structures.

\subsection{Why You Should Choose OPAL}

There are many other libraries that provide algorithms or data structures, but most of them are either too complicated or contain too few or no helpful comments at all. First and foremost, OPAL aims to provide an easy to understand implementation built on good coding practices. Every algorithm and structure in OPAL is implemented using modern C++ template techniques. This allows for vast flexibility in use and makes it easier to extend the library with new functionality.

\subsection{The Purpose of OPAL}

OPAL is a project that I started in order to build a framework that would help me easily create algorithm implementations for lectures at my university. I wanted to refresh my knowledge of modern C++, and this is why I decided to use it as the core programming language in OPAL. I hope that OPAL will be useful to other people as well, and I will be happy to receive any feedback or suggestions.
